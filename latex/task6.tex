\section{Task 6 Wide Halo Transfers}

\subsection{Implementation}
The implementation of wide halo method is quite straightforward. Update the halo with $w$ width every $w$ simulation time steps. 
For each time step in a cycle of total $w$, the part of field to be upated decreases by two.
Therefore, at time step $0$, the size of that part is $(M_{loc}+2*w-2)(N_{loc}+2*w-2)$, at the step $i$, the size is 
$(M_{loc}+2w-2-2i)(N_{loc}+2*w-2-2i)$.
Again, similar to Task 4, in each communication,
we collect $w$ rows/columns into a temporary buffer firstly and then send/receive in one time. This will reduce the $t_s$ term
in communication cost.

\subsection{Performance Model}
There is a trade off in this approach. The startup time $t_s$ part of communication time is decreased by a factor of $w$. However,
the computation cost increases, since one has to do more computation for additional part of field to be updated.

In derivation of the following performance model, we assume that simulation time cycle $r$ is much larger than $w$,
since if it is not, the wide halo method will be worse than the method without wide halo, which is not our interest.
Furthermore, for simplicity, we assume that $r \% w = 0$.

The total time cost is 
\[
	t_{\textrm{total}} = \frac{r}{w} t_{\textrm{cycle}}
\]
where $t_{\textrm{cycle}}= t_{\textrm{comp}} + t_{\textrm{comm}}$ is the cost in one cycle ($w$) of time steps.

For the communication cost per cycle
\[
	t_{\textrm{comm}} = 2 \left[ 2 t_s + w\cdot \left( N_{loc} + M_{loc} + 2 w \right) t_w  \right]
\]

For the computation cost per cycle
\begin{eqnarray*}
	t_{\textrm{comm}} &=& \sum^{w-1}_{0} (M_{loc}+2w-2-2i)(N_{loc}+2*w-2-2i) t_f \\
	&=& w M_{loc} N_{loc} t_f + w(w-1) (M_{loc}+N_{loc}) t_f + \frac{2}{3} w (w-1) (2w-1) t_f
\end{eqnarray*}
where the summation over $i$ reflects the part of field to be updated changes in each time step.

Collecing all the terms, the final total cost is
\begin{equation} \label{equ4}
	t_{\textrm{total}} = r \left[ 
		\frac{4}{w} t_s + 2 \left( M_{loc} + N_{loc} + 2 w \right) t_w +
		M_{loc} N_{loc} t_f + (w-1) (M_{loc}+N_{loc}) t_f + \frac{2}{3}(w-1) (2w-1) t_f
		\right]
\end{equation}
where $N_{loc} = N/Q$ and $M_{loc} = M/Q$.

Comparing wide halo Eq \ref{equ4} with non-wide halo Eq \ref{equ2}, there are two differences.
The first one is the startup term in communication cost decreases by a factor of $w$, since $t_s$ is
a relative large term, so this reduction is not negligible. The second one is the additional computation
cost, which is proportional to $w$ or $w^2$, if $w$ is not small and comparable to $N_loc$ or $M_loc$,
this term is comparable to the non-wide halo computation term $M_{loc} N_{loc} t_f$. Therefore, one 
would be better not to choose a large $w$.

Consequently, in the case where the communication cost is not negligible, the wide halo will have 
a better performance due to the reduction of startup communication $t_s$ term.

\subsection{Experiments}
According to the argument in previous subsection, we should choose the case in which communication cost
is large to do experiments.
Hence we choose $P\neq Q$ and $M\neq N$ from the aspect ratio argument in Task 4. 

In order to investigate the effect of wide halo method, we choose $P=4$, $Q=8$, $N=1\times10^4$, $M=2\times10^4$, 
$r=100$, and vary $w = \{1,5,10,500\}$.  
%In the second set of experiments, we vary $r=\{1,10,10^2,10^3\}$ and keep $w=5$. 
I choose a large $w$ to show that when $w$ is large the total cost will increase due to the additional computation cost.
The results are shown in Tab \ref{tab7}. The case of $w=1$ is the one without wide halo effect, we can 
see that when $w$ increases (except the case $w=500$), the total time decreases, so the wide halo method is effective. 
Furthermore, when $w=500$, the total cost increases by almost a factor of $2$. This is because when $w=500$, which
is near $N_{loc} = 10^4/8 = 1250$, the additoinal computation cost is comparable to the original one.

\begin{table}[h]
	\centering
	\caption{Wide halo effect. Experiment Set 1, varying $w$ and keep $r, P, Q, M, N$ fixed.}
	\label{tab7}
	\begin{tabular}{lllll}
		\hline
		w               & $1$ & $5$ & $10$ & $500$ \\ \hline
		time            & $8.65$s & $8.52$s & $8.21$s & $16.7$s  \\ \hline
	\end{tabular}
\end{table}

\subsection{Comparison with Overlapping Method}
The wide halo method will reduce the startup communication cost, the $t_s$ term.  
The overlapping method will also reduce the communication cost, however, it reduce the 
data transfering term, the $t_w$ term, by overlapping it with computation.

Consequently, both methods will reduce the communication cost, but from different aspects.
