\section{Task 7 Literature Rewiew on Cache Oblivious}

In following paragraphs, we will first introduce cache oblivious algorithm,
and then discuss its application and impacts in stencil computation.

\vspace{\baselineskip}

Reduction of cache miss is one of important aspect of algorithm optimization.
It is also true in parallel computation, which highly emphasizes on the performance of algorithm.
Many programmers is aware of this fact and implement their algorithm with the consideration of cache miss.
However, many of such algorithms will depend on the parameters of cache, such as cache line size and the number of cache line.
These cache parameters are actually relevant to hardware, which is out of control of programmer.
Consequently, one program optimized with some specific cache parameter may have very low performance on a machine with another set of cache parameters. 
Furthermore, in modern computers, there are more than one level of caches with difference cache parameters, which make the above issues more serious.
To solve this problem, Frigo and his collabrators \cite{Frigo1999} proposed the idea of cache oblivious algorithm for tall cache in parallel programming. 
Their cache oblivious algorithm doesn't know the cache parameters, 
but uses caches as effeciently as those algorithms who knows cache parameter foreahead.

In the master thesis of Prokop \cite{Prokop1999} applied the cache oblivious method to stencil problem for the first time.
But his algorithm can only work for one dimension space problem with square spacetime region.
In 2005, Frigo proposed an algorithm which can work for arbitrary stencil and arbitrary spacetime dimension \cite{Frigo2005}.
In this paper, the spacetime concerned in the problem is divided into \textit{trapezoidal regions}.
If the flow velocity is zero, the the trapezodial regions become rectangular regions.
Once the the field value $u$ at the bottom edge(or surface or cube, depending on the space dimension) of the trapezodial region is known, 
the whole $u$ values in the trapezodial region can be obtained by applying the approciate fluid equations.
These trapezodial regions can be divided into smaller trapezodial region with nearly equal size.
This dividing process can be done iteratively until each trapezodial region is small enough to entirely fit in cache.
Therefore, in this method, the field values are not calculated in the order of their spatial index any longer, 
but in the order of trapezodial region index.
Frigo proofs this method is cache efficient \cite{Frigo2005}.

Making use of this method, cache miss is reduced. For example, Frigo \cite{Frigo2007} applied this method to heat diffusion problem.
For two dimension problem, the cache miss will be reduced by a factor of 5 on average, 
and for one dimension heat diffusion problem, the cache miss will be reduced to $1\%$ \cite{Frigo2007},
Therefore, the cache oblivious method improves the performance of parallel stencil problem by reducing the cache miss.



