\section{Task 2 The Effect of Non-Blocking Communication}

I add compilation option $BLOCK$ to make a selection on which kind of communication to be used.

\begin{lstlisting}[language=c]
# ifdef BLOCK
   // blocking communication
# endif
# ifndef BLOCK
  // non-blocking communication
# endif
\end{lstlisting}

Since at current stage, we restrict $P=1$, so there are \textit{only} left-right communications. In this case,
The communication time $t_{\textrm{comm}}$ and computation time $t_{\textrm{comp}}$ should be
\begin{eqnarray*}
	t_{\textrm{comm}} & \propto & 2\times \left( t_s + M_{loc}\cdot t_w  \right) \cdot r \propto M \cdot r \\
	t_{\textrm{comp}} & \propto & N_{loc} \cdot M_{loc} \cdot r \propto \frac{M\cdot N}{Q}\cdot r
\end{eqnarray*}
where $r$ is the number of time steps in simulation, $N_{loc}, M_{loc}, M, N, Q$ have the same meaning as in the 
assignment specification, and we have use the fact $M_{loc}=M, N_{loc} = N/Q$. Obviously, communication time
is independent of $Q$, since there is no top-bottom communication in this case. Consequently, the parameters
to maximize the impact of communication should be:

\begin{itemize}
	\item increase $r$.
	\item use large $M$.
	\item large $Q$ to reduce the percentage of time used by computation. Furthermore, large $Q$ will result in large 
		$nproc$, so that there will be large mount of communication between different nodes, rather than between
		processors in a socket only. This is a hardware factor to increase communication time.
	\item use small $N$.
\end{itemize}

So I choose parameters $nprocs = 32, r = 100, N = 96$ and a varying $M = 2\times10^2, 2\times 10^3, 2\times10^4, 2\times10^5$. 
The results can be found in Tab \ref{tab1}.

\begin{table}[h]
	\centering
	\caption{The performance of blocking and non-blocking communication}
	\label{tab1}
	\begin{tabular}{lllll}
		\hline
		M            & $2\times 10^2$ & $2\times 10^3$ & $2\times 10^4$ & $2\times 10^5$ \\ \hline
		blocking     & $1.45\times 10^{-2}$s &  $2.26\times 10^{-2}$s &   $9.96\times 10^{-2}$s &  $1.32$s \\
		non-blocking & $6.66\times 10^{-2}$s & $4.34\times 10^{-2}$s & $8.48\times 10^{-2}$s & $1.20$s \\ \hline
	\end{tabular}
\end{table}

From the results in Tab \ref{tab1}, for large $M$, non-blocking's performance is better than that of blocking. While, for
small value of $M$, blocking communication is better. However, in a real fluid dynamics application, the number of 
grid usually is very large due to the multiple scale nature of the fluid problem. For example, in a fluid model with
convection and diffusion, the number of grid should be larger than the $\frac{\textrm{simulation scale}}{\textrm{diffusion scale}}$.
Consequently, we will adopt non-blocking communication in the following.

Therefore, I comment out the blocking communication code.
